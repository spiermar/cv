%!TEX TS-program = xelatex
\documentclass[]{mspier-cv}
\addbibresource{bibliography.bib}

\begin{document}
\header{martin}{spier}
       {performance engineer at Netflix}

% In the aside, each new line forces a line break
\begin{aside}
  \section{about}
    {Bay Area, CA}
    {+1 (425) 351-8800}
    \href{mailto:hello@martinspier.io}{hello@martinspier.io}
    \href{http://martinspier.io}{martinspier.io}
    \href{https://github.com/spiermar}{github://spiermar}
    \href{https://twitter.com/spiermar}{twitter://@spiermar}
    \href{https://www.linkedin.com/in/martinspier}{linkedin://martinspier}
  \section{skills}
    performance engineering
    large scale systems
    scalability
    systems architecture
    cloud computing
    (aws)
    distributed systems
    java/jvm tuning
    performance testing
    (jmeter, gatling, loadrunner)
    *nix, linux
    tomcat, apache httpd
    data analysis
    (hive, presto, ipython, pandas, tableau)
  \section{programming}
    python
    javascript
    (node.js, es6)
    java, groovy
  \section{languages}
    english
    portuguese
\end{aside}

\section{experience}

\experienceitem
  {Senior Performance Engineer}
  {Netflix}
  {Dec 2012 - Present}
  {Los Gatos, CA}
  {
    Optimize service reliability and performance, and help scale Netflix's high-traffic, large-scale distributed systems.
    Thorough performance analysis and tuning across all services and layers.
    Find optimizations both within application stacks and across the infrastructure.
    Develop effective observability tooling and and assist with production triage and root cause analysis on performance or availability issues.
    \newline
    Created Vector, Netflix's open source on-host, high-resolution performance monitoring framework.
    Drove a company-wide program to improve service availability by implementing and applying a collection of best practices related to the development, deployment, and operation of cloud services.
    Developed Mogul, a bottleneck analysis tool that inspects internal and downstream dependency demand for services.
    Developed Slalom, a high-level demand analysis tool that helps visualize demand flow on large scale systems.
    Created a performance trend report to help identify long-term performance regression.
    Implemented and extended the capabilities of the in-house performance testing framework.
    Developed and supported the performance analysis tool with fully automated analysis capabilities.
    Drove performance test adoption company-wide and integration on continuous integration environments.
  }

\experienceitem
  {Performance Engineer}
  {Expedia}
  {Jan 2011 - Dec 2012}
  {Bellevue, WA}
  {
    Responsible for ensuring the performance, stability and scalability of large scale platforms that support the Hotels, Flights, Cars and Ads LOBs.
    Led the performance engineering effort during the first two full LOB migrations to the new Java platform based on Tomcat, Hotels and Flights, and also the Ads engine, DoubleClick, to Linux.
    Improved early issue discovery and quality cost by introducing performance tests to the Continuous Delivery pipeline using HP Performance Center, Jenkins and in-house developed tools.
    Promoted awareness about performance related issues by introducing an executive dashboard and alerting system for Trunk performance test executions.
    Deployed and maintained the client-side performance evaluation tool based on the open-source project WebPagetest.
    Introduced JVM MBean and Linux monitoring to performance tests through HP SiteScope and Java code profiling with YourKit.
    Introduced the concept of Java heap trend analysis by linear regression.
    Contributed to the team’s process standardization by developing test plan and report standards, and introduced an environment reservation system.
    Supported performance test execution by non-performance engineering resources, including functional testers and developers, and coached offshore resources in China.
  }

\experienceitem
  {Performance Engineer}
  {Dell}
  {Nov 2007 - Jan 2011}
  {Porto Alegre, Brazil}
  {
    Led a team of 4 performance engineers responsible for one of the most critical Dell business streams, working mostly on large and global projects, with 50+ critical applications in scope.
    Responsible for the team’s development, capacity planning, project allocation and structure definition.
    Led the performance research group in conjunction with a local university, focusing on performance of virtualized environments, performance modeling and performance engineering tools.
    Coordinated the alignment between cross functional teams, including development, functional test, architecture and infrastructure, spread across multiple locations  in the USA, India, UK, France, Russia, Malaysia, Singapore and Japan.
    Led the performance test efforts for the largest IT project in 2009 and 2010 that aimed to replace Dell’s worldwide order management system and quoting engine.
    Contributed significantly to the team’s process standardization initiative by creating standards for plans and reports and defining the standard life cycle for performance engineering projects.
    Specialized on several technologies, including Oracle EBS, Oracle WebLogic, Citrix, Microsoft IIS, Web Services, IBM MQ, .NET Framework, Oracle Database and Microsoft SQL Server.
  }

\experienceitem
  {Software Test Analyst}
  {Dell}
  {Jan 2007 - Nov 2007}
  {Porto Alegre, Brazil}
  {
    Responsible for all functional testing on 3 large scale applications, owning all related tasks and artifacts, including test plans, schedule and reports.
    Acted as Quality Center Administrator for the whole portfolio, managing all user groups, rights and test processes, providing training on the use of the system and defect tracking procedure to testers and developers.
    Developed a tool to generate test data automatically based on templates and input files, reducing test effort by 84\% across the team.
    Developed a solution to automate the scenarios creation on Quality Center.
    Designed an application Wiki, a centralized space to store information about applications in scope and facilitate the knowledge transfer between team members.
    Worked directly with the internal customers, defining test scope, schedule and test plans.
    Received two awards for excellence on leadership, agility and technical skills.
  }

\experienceitem
  {Software Test Analyst}
  {CPM Braxis}
  {Sep 2006 - Jan 2007}
  {Porto Alegre, Brazil}
  {
    Contractor position for Dell Inc. at one of the largest IT Service providers in Brazil, with over 5.4 thousand employees and a large outsourcing unit.
    Part of Dell’s first major project in the Brazil development center, with benefits surpassing U\$3.842.294 (estimated for one fiscal year).
    Led all functional test efforts for 3 out of 10 applications in scope and had 0\% defect leakage to production, automating all test cases using HP QTP.
  }

\section{education}

\begin{entrylist}
  \entry
    {2009 - 2012}
    {Specialization, Project Management}
    {Porto Alegre, Brazil}
    {Pontifícia Universidade Católica do Rio Grande do Sul}
  \entry
    {2002 - 2008}
    {B.Sc. Computer Science}
    {Porto Alegre, Brazil}
    {Pontifícia Universidade Católica do Rio Grande do Sul}
\end{entrylist}

\section{opensource}

\opensourceitem
  {Vector}
  {Project Lead}
  {\href{https://github.com/Netflix/vector}{github.com/Netflix/vector}}
  {An on-host performance monitoring framework which exposes hand picked high resolution metrics to every engineer’s browser.}

\opensourceitem
  {D3.js Flame Graph}
  {Project Lead}
  {\href{https://github.com/spiermar/d3-flame-graph}{github.com/spiermar/d3-flame-graph}}
  {A D3.js plugin that produces flame graphs from hierarchical data.}

\section{publications}

\publicationitem
  {Log-Based Approach for Performance Requirements Elicitation and Prioritization}
  {Sep 2012}
  {Requirements Engineering Conference (RE), 2012 20th IEEE International}
  {Link to article: \url{http://goo.gl/ufDDyO}}

\publicationitem
  {Development of a Parametric Model to Estimate Software Development Effort}
  {Jun 2008}
  {EDIPUCRS}
  {Link to article: \url{http://goo.gl/Q1B0k7}}

%%% This piece of code has been commented by Karol Kozioł due to biblatex errors.
%
%\printbibsection{article}{article in peer-reviewed journal}
%\begin{refsection}
%  \nocite{*}
%  \printbibliography[sorting=chronological, type=inproceedings, title={international peer-reviewed conferences/proceedings}, notkeyword={france}, heading=subbibliography]
%\end{refsection}
%\begin{refsection}
%  \nocite{*}
%  \printbibliography[sorting=chronological, type=inproceedings, title={local peer-reviewed conferences/proceedings}, keyword={france}, heading=subbibliography]
%\end{refsection}
%\printbibsection{misc}{other publications}
%\printbibsection{report}{research reports}

\end{document}
