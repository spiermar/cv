\section{experience}

\begin{entrylist}
  \entry
    {2012-Present}
    {Performance Architect}
    {Los Gatos, CA}
    {
      \textbf{Netflix}
      \begin{itemize}
        \item Optimize service reliability and performance, and help scale Netflix's high-traffic, large-scale distributed systems.
        \item Thorough performance analysis and tuning across all services and layers.
        \item Find optimizations both within application stacks and across the infrastructure.
        \item Develop effective observability tooling and and assist with production triage and root cause analysis on performance or availability issues.  
      \end{itemize}
      \begin{itemize}
        \item Developed Icarus, Netflix's real user performance monitoring solution, that runs on every user device, processes over 180B+ events and petabytes of data every day, provides an intuitive GUI for analysis, alerting and anomaly detection.
        \item Creted multiple d3.js based visualizations for performance data, including plugins for flame graphs and heatmaps.
        \item Developed FlameScope (\href{https://github.com/Netflix/flamescope}{github.com/Netflix/flamescope}), a visualization tool for exploring different time ranges as flame graphs.
        \item Created FlameCommander, a cloud profiling tool that allows engineers to capture and analyze CPU, memory allocation profiles, heapdumps and more, on any cloud instance or container, at a click of a button.
        \item Created Vector (\href{https://github.com/Netflix/vector}{github.com/Netflix/vector}), Netflix's open source on-host, high-resolution performance monitoring framework.
        \item Drove a company-wide program to improve service availability by implementing and applying a collection of best practices related to the development, deployment, and operation of cloud services.
        \item Developed Mogul, a bottleneck analysis tool that inspects internal and downstream dependency demand for services.
        \item Developed Slalom, a high-level demand analysis tool that helps visualize demand flow on large scale systems.
        % \item Created a performance trend report to help identify long-term performance regression.
        \item Implemented and extended the capabilities of the in-house performance testing framework.
        \item Developed and supported the performance analysis tool with fully automated analysis capabilities.
        % \item Drove performance test adoption company-wide and integration on continuous integration environments.
      \end{itemize}
    }
  % monahees+
  % codenation
  % handson
  \entry
    {2011-2012}
    {Performance Engineer}
    {Bellevue, WA}
    {
      \textbf{Expedia}
      \begin{itemize}
        \item Responsible for ensuring the performance, stability and scalability of large scale platforms that support Expedia's Hotels, Flights, Cars and Ads lines of business.
        \item Led the performance engineering efforts during the large platform migration from .NET to Java/Tomcat.
        \item Led the performance engineering efforts during the Ads platform migration to Linux.
        \item Improved early issue dectection by introducing performance tests to the Continuous Delivery pipeline using HP Performance Center, Jenkins and in-house developed tools.
        % \item Created the executive performance dashboard and alerting system, raising awareness about performance issues.
        \item Deployed and maintained the client-side performance evaluation tool based on the open-source project WebPageTest.
        \item Introduced JVM and Linux monitoring to the tools suite using HP SiteScope.
        \item Introduced Java code profiling to the tools suite using YourKit.
        \item Introduced the concept of Java heap trend analysis by linear regression and fully automated the process.
        % \item Contributed to the team’s process standardization by developing test plan and report standards.
        % \item Created the performance test environment reservation system.
        % \item Supported performance-related tasks by other engineering teams, QA and product developmen teams.
        % \item Coached offshore resources in China.
      \end{itemize}
    }
  \entry
    {2007-2011}
    {Performance Engineer}
    {Porto Alegre, Brazil}
    {
      \textbf{Dell}
      \begin{itemize}
        \item Led a team of 4 performance engineers responsible for one of the most critical Dell business streams, global order management, working mostly on large, global projects, with 50+ critical applications in scope.
        \item Responsible for the team's development, capacity planning, project allocation and structure definition.
        \item Led the performance research group in conjunction with a local university, focusing on performance of virtualized environments, performance modeling and performance engineering tools.
        \item Coordinated the alignment between cross functional teams, including product development, QA, architecture and infrastructure, spread across multiple locations in the USA, India, UK, France, Russia, Malaysia, Singapore and Japan.
        \item Led the performance engineering efforts for the largest IT project in 2009 and 2010, which aimed to replace Dell's worldwide order management system and quoting engine.
        % \item Contributed significantly to the team’s process standardization initiative by creating standards for plans and reports and defining the standard life cycle for performance engineering projects.
        % \item Specialized on several technologies, including Oracle EBS, Oracle WebLogic, Citrix, Microsoft IIS, Web Services, IBM MQ, .NET, Oracle Database and Microsoft SQL Server.
      \end{itemize}
    }
  \entry
    {2006-2007}
    {Software Test Analyst}
    {Porto Alegre, Brazil}
    {
      \textbf{Dell}
      \newline
      Responsible for all QA efforts on three large-scale applications. Acted as Quality Center Administrator for the whole portfolio. Developed multiple tools to automate repetitive tasks and reduce effort. Received two awards for excellence on leadership, agility and technical skills.
      % \begin{itemize}
      %   \item Responsible for all QA efforts for three large-scale applications, owning all related tasks and artifacts, including test plans, schedule and reports.
      %   \item Acted as Quality Center Administrator for the whole portfolio, managing all user groups, rights and test processes, providing training on the use of the system and defect tracking procedure to testers and developers.
      %   \item Developed a tool to generate test data automatically based on templates and input files, reducing test effort by 84\% across the team.
      %   \item Developed a solution to automate the scenario creation on Quality Center.
      %   \item Designed an application Wiki, a centralized space to store information about applications in scope and facilitate the knowledge transfer between team members.
      %   \item Worked directly with the internal customers, defining test scope, schedule and test plans.
      %   \item Received two awards for excellence on leadership, agility and technical skills.
      %   \item Led the QA efforts during the first major project of the Brazilian development unit, with benefits surpassing U\$3.842.294, having all functional testing automated, and ending with 0\% defect leakage to production.
      % \end{itemize}
    }
\end{entrylist}
